\date{$fecha$}
\author{$autor$}
\title{$titulo$}

\begin{document}
    \begin{titlepage}
        \begin{center}
            \vspace*{-1in}
            \begin{figure}[htb]
                \begin{center}
                    \includegraphics[width=3cm]{./latex/img/logo}
                \end{center}
            \end{figure}
            \begin{large}
                \textbf{Universidad de Valladolid}
            \end{large}

            \vspace*{0.15in}
            \vspace*{0.6in}
            \begin{large}
                \textbf{ESCUELA DE INGENIERÍA INFORMÁTICA}
            \end{large}
            \vspace*{0.2in}
            \textbf{ GRADO EN INGENIERÍA INFORMÁTICA}\\
            \textbf{ MENCIÓN EN $mencion$ }
            \vspace*{0.5in}
            \rule{140mm}{0.1mm}\\
            \vspace*{0.3in}
            \begin{large}
                \textbf{{\LARGE $titulo$\\}}
            \end{large}
            \vspace*{0.6in}
            \rule{140mm}{0.1mm}\\
            \vspace*{2in}
            \begin{large}
                \begin{flushright}
                    \textbf{Alumno/a: $autor$ \\
                    \vspace*{0.3in}
                    Tutores/as: }
                \end{flushright}
            \end{large}
        \end{center}
    \end{titlepage}

    \newpage
    \mbox{}	
    \thispagestyle{empty} % para que no se numere esta página

    \chapter*{}
    \pagenumbering{Roman} % para comenzar la numeración de paginas en números romanos

    \begin{flushright}
        \textit{%Dedicatoria,\\
        $dedicatoria$}
    \end{flushright}

    \chapter*{Agradecimientos} % si no queremos que añada la palabra "Capitulo"
    \addcontentsline{toc}{chapter}{Agradecimientos} % si queremos que aparezca en el índice
    \markboth{AGRADECIMIENTOS}{AGRADECIMIENTOS} % encabezado 

    $agradecimientos$

    \chapter*{Resumen} % si no queremos que añada la palabra "Capitulo"
    \addcontentsline{toc}{chapter}{Resumen} % si queremos que aparezca en el índice
    \markboth{RESUMEN}{RESUMEN} % encabezado
    \begin{flushleft}

    $resumen$

    \end{flushleft}


    \chapter*{Abstract} % si no queremos que añada la palabra "Capitulo"
    \addcontentsline{toc}{chapter}{Abstract} % si queremos que aparezca en el índice
    \markboth{ABSTRACT}{ABSTRACT} % encabezado
    \begin{flushleft}

    $abstract$

    \end{flushleft}

    \tableofcontents % indice de contenidos

    \cleardoublepage
    \addcontentsline{toc}{chapter}{Lista de figuras} % para que aparezca en el indice de contenidos
    \listoffigures % indice de figuras

    \cleardoublepage
    \addcontentsline{toc}{chapter}{Lista de tablas} % para que aparezca en el indice de contenidos
    \listoftables % indice de tablas

    $body$

    \cleardoublepage
    \addcontentsline{toc}{chapter}{Bibliografía}
    %\renewcommand\bibname{Referencias Web}

    %\begin{thebibliography}{X}
    %    \bibitem{ref1} \textit{Ejemplo}, \\
    %    \textsc{ejemplo.com}.
    %    \\Recuperado a tal fecha, \\de \href{http://ejemplo.com}
    %\end{thebibliography}
\end{document}
