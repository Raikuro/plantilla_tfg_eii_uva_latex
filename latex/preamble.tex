\documentclass[openright,twoside,10pt]{book}
\usepackage[b5paper,left=2cm,top=2.5cm,right=1.5cm,bottom=2.5cm]{geometry} 
\usepackage[spanish]{babel} % espanol
\usepackage[utf8]{inputenc} % acentos sin codigo
\usepackage{graphicx} % gráficos
\usepackage{lscape}
\usepackage{fancyvrb}
\usepackage{fancyhdr}
\usepackage{wrapfig}
\usepackage[hidelinks]{hyperref}
\usepackage{biblatex}
\bibliography{bibliografia}

\providecommand{\tightlist}{%
  \setlength{\itemsep}{0pt}\setlength{\parskip}{0pt}}
  
\setlength{\parskip}{10pt plus 1pt minus 1pt}
 % aqui definimos el encabezado de las paginas pares e impares.
\rhead[]{}

\renewcommand{\headrulewidth}{0.5pt}

% aqui definimos el pie de pagina de las paginas pares e impares.
\rfoot[\thepage]{\thepage}
\cfoot[]{}
\renewcommand{\footrulewidth}{0pt}

%redefino el verbatim
%\renewenvironment{verbatim}{\begin{Verbatim}[frame=single,fontsize=\small]}{\end{Verbatim}}


% aqui definimos el encabezado y pie de pagina de la pagina inicial de un capitulo.
\fancypagestyle{plain}{
\fancyhead[R]{}
\fancyfoot[C]{}
\fancyfoot[R]{\thepage}
\renewcommand{\headrulewidth}{0.5pt}
\renewcommand{\footrulewidth}{0pt}
}

\pagestyle{fancy} % seleccionamos un estilo

